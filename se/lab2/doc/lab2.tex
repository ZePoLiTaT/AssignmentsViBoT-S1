\documentclass{article}


 % Required for inserting code snippets
\usepackage{mathtools}
\usepackage{graphicx}
\usepackage{subfig}
\usepackage{verbatim}
\usepackage{algpseudocode}
\usepackage{natbib}
\usepackage{url}
\usepackage{listings}
\usepackage{float}
\usepackage{array}
\usepackage{booktabs}
\usepackage{amssymb}
%\setcounter{MaxMatrixCols}{16}

%%%%%%%%%%%%%%%%%%%%%%%% PAINTING LIBRARY
\usepackage{pgf}
\usepackage{tikz}
\usetikzlibrary{arrows,automata}
\usepackage[latin1]{inputenc}
\usepackage[upright]{fourier}
\usetikzlibrary{matrix}
\usepackage{fullpage,amsmath}
\usepackage{times}
\usepackage{geometry}
\usetikzlibrary{mindmap,backgrounds}

%%%%%%%%%%%%%%%%%%%%%%%% THNX TO OZAN FOR THE CODE HEADER ;)
\usepackage{color}
\usepackage{xcolor}
\usepackage{listings}

\usepackage{courier}
\definecolor{DarkGray}{rgb}{0.43,0.35,0.35} % Comment color
\lstset{
	language=C++,								% choose the language of the code
		basicstyle=\footnotesize\ttfamily,  % the size of the fonts that are used for the code
		numbers= left,						% where to put the line-numbers
		numberstyle=\tiny,					% the size of the fonts that are used for the line-numbers
		stepnumber=1,						% the step between two line-numbers. If it is 1 each line will be numbered
		numbersep=10pt,						% how far the line-numbers are from the code
		backgroundcolor=\color{white},		% choose the background color. You must add \usepackage{color}
		keywordstyle=\color{DarkGray}\bf,
	showspaces=false,						% show spaces adding particular underscores
		showstringspaces=false,				% underline spaces within strings
		showtabs=false,						% show tabs within strings adding particular underscores
		keywordstyle=\color{DarkGray}\bf,
		stringstyle=\color[rgb]{0.627,0.126,0.941},
		xleftmargin=17pt,
		framexleftmargin=17pt,
		framexrightmargin=5pt,
		framexbottommargin=4pt,
		%frame=b,         
		frame=single,					% adds a frame around the code
		tabsize=2,						% sets default tabsize to 2 spaces
		captionpos=t,					% sets the caption-position to bottom (t=top, b=bottom)
		breaklines=true,				% sets automatic line breaking
		breakatwhitespace=false,		% sets if automatic breaks should only happen at whitespace
		escapeinside={\%*}{*)}          % if you want to add a comment within your code
}

\usepackage{caption}
\DeclareCaptionFont{white}{\color{white}}
\DeclareCaptionFormat{listing}
{
	\colorbox[rgb]{0.83, 0.85, 0.88}
	{\parbox{\dimexpr\textwidth-8\fboxsep\relax}{#1#2#3}}
}

\captionsetup[lstlisting]
{
	format=listing,
		labelfont=white,
		textfont=white, 
		singlelinecheck=false, 
		margin=0pt, 
		font={bf,footnotesize}
}

\usepackage{textcomp}

%%%%%%%%%%%%% WATERMARKS  %%%%%%%%%%%%%%%%%
%\usepackage{eso-pic}
%\AddToShipoutPicture{
%    \includegraphics[width=1.74\textwidth,natwidth=1223,natheight=121]{figs/lowImg2.png}}

\DeclarePairedDelimiter\abs{\lvert}{\rvert}


%%%%%%%%%%%%% PARAGRAPHS %%%%%%%%%%%%%%%%%
%No ident in new paragraph
\usepackage[parfill]{parskip}
%% Add a space of 1.5 lines between paragraphs
\parskip=1.1\baselineskip

%%%%%%%%%%%%% DEFAULT FONT %%%%%%%%%%%%%%%
%Sans-serif font by default
%\renewcommand{\familydefault}{\sfdefault}
%\usepackage{mathptmx}

%\usepackage{bookman}
%%%%\usepackage[default]{droidserif}
%\usepackage[T1]{fontenc}
%\usepackage{lxfonts}
%\usepackage{tgadventor}
%\renewcommand*\familydefault{\sfdefault} %% Only if the base font of the document is to be sans serif
%\usepackage[T1]{fontenc}


\newcommand{\specialcell}[2][c]{%
  \begin{tabular}[#1]{@{}c@{}}#2\end{tabular}}

\usepackage{float}



\begin{document}
\title{Report Lab 2}
\date {\today}
\author{Tatiana Lopez Guevara}
\maketitle

\section{Input/Output}
In this section I played with the operators that 
handle the input and output. 

One thing to notice here
was that when I used \verb+cin>>+\textit{string\_var} 
and the user input was composed of several words separated
by space, only the first string token was read, so if I want
to read a whole line, I have to use the \verb+getline+ function.

Also, the other characters stayed in the buffer and therefore
the next command that reads from the standard input like \verb+getline+ 
would take this value. To solve this, the command \verb+cin.ignore(INT_MAX, '\n')+
was used. This clears max possible characters in the buffer or until it 
reaches an enter character \cite{stack1}.

It is also important to use \verb+cin.clear()+ to clear the error flag
after reading, otherwise, if by some reason the user inputs a wrong
type of data, then the error flag will activate and will ignore
the next readings after it.

\begin{lstlisting}[label=lst_inout,caption=Input]
/**
 * @brief exampleInputOutput illustrates the use of the
 * standard input and output commands and some tricks
 * like cin.ignore()
 */
void exampleInputOutput()
{
	string line;

	cout<<"Introducing the one and unque COUT function!"<<endl;
	cout<<"Please write a word and I'll read it with CIN: "<<endl;
	cin>>line;

	//This is needed because we want to flush non used characters
	cin.ignore(INT_MAX, '\n');

	cout<<"Ok, You entered "<<line<<" with CIN and I've displayed it using COUT. Magic!"<<endl;

	cout<<"Now, I want to test getline(). Please enter something else: "<<endl;
	getline(cin, line);
	cout<<"You entered: "<<line<<endl;
}
\end{lstlisting} 

\section{Parameters}
Three diferent functions to swap were implemented to test the value
or reference parameters and its effect on the affected variables.
The first version of it (\verb+swap_1+) 
receives the parameters by value, and therefore
any change that is done inside the function does not prevail after 
the function ends. In this case this means that the function does
not work correctly for swapping.

However, the second version of \verb+swap_2+ and \verb+swap_3+  
receive the parameters
by reference and pointers to the variables respectively
and therefore the modifications are correctly preserved
after the function finishes. The difference between these 2 calls
is that in the last case you need to provide the address of the 
variable in the main by using the \verb+&+ operator.

Also, the declaration of the pointer version
was changed a little by specifying that the parameter was a constant 
pointer to a variable integer \cite{const}:

\verb+void swap_3(int* const, int* const)+ 

\begin{lstlisting}[label=lst_param,caption=Swap]
/**
 * @brief swap_1 two values that are passed by value
 * This works fine inside the function, but the changes
 * do not persist after
 * @param a first value
 * @param b second value to swap
 */
void swap_1(int a,int b)
{
	int tmp;

	tmp = a;
	a = b;
	b = tmp;
	cout<<" --> Internal result swap_1 function: "<<a<<", "<<b<<endl;
}

/**
 * @brief swap_2 two values that are passed by reference
 * @param a first value
 * @param b second value to swap
 */
void swap_2(int &a,int &b)
{
	int tmp;

	tmp = a;
	a = b;
	b = tmp;

	cout<<" --> Internal result swap_2 function: "<<a<<", "<<b<<endl;
}

/**
 * @brief swap_3 two values that are pointers
 * @param a pointer to the first value
 * @param b pointer to the second value to swap
 */
void swap_3(int* const a, int* const b)
{
	int tmp;

	tmp = *a;
	*a = *b;
	*b = tmp;

	cout<<" --> Internal result swap_3 function: "<<a<<", "<<b<<endl;
}
\end{lstlisting} 

\section{Mutiple returns}
C++ does not allow multiple return values.
In order to implement the function \verb+CartesianToPolar+, 
four parameters were needed:
the first two provided the \verb+a, b+ input constant values and the last
two were used to return the $\rho$ and $\theta$.

The difference between the functions for calculating the angle 
between \verb+a, b+ is that \verb+atan2+ takes into account the
sign of both variables (since it receives 2 parameters) 
to determine the quadrant of the angle
whereas \verb+atan+ only receives one parameter ($\frac{b}{a}$)
and if this term is positive, the function doesn't know if the 
angle was in the first or the third quadrant \cite{atan}.

\begin{lstlisting}[label=lst_mult,caption=Cartesian To Polar]
/**
 * @brief cartesianToPolar returns the norm and angle from the elements
 * z = a + ib of a complex number
 * @param a real part
 * @param b imaginary part
 * @param norm returned value with the norm
 * @param angle returned value with the angle
 */
void cartesianToPolar(const double &a, const double &b, double &norm, double &angle)
{
	norm = sqrt(a*a + b*b);
	angle = atan2(b,a);
}

\end{lstlisting} 

\section{Default parameter}
The most important things to remark with the \verb+isMultipleOf+
function was that the default value was especified in the
prototype of the header and not on the source code or cpp file.
Also, that there was no need to have the name of the 
default variable written, so in the end I had this declaration:

\verb+bool isMultipleOf(const int &, const int=2);+ 

\begin{lstlisting}[label=lbl_def,caption=Multiple Of]
/**
 * @brief isMultipleOf evaluates if a number p is multiple of another (q).
 * If no q is provided it returns if p is even.
 * @param p number we want to know if it is multiple of another
 * @param q number we want to evaluate p to see if it is multiple of it
 * @return true if it is multiple, false otherwise
 */
bool isMultipleOf(const int &p, const int q)
{
	return p%q==0;
}
\end{lstlisting} 

\section{Recursivity}

Here I created a function that determines if a given number
\verb+p+  is prime or not by validating recursively that it is only 
divisible by 1 and by itself. 

A wrapper function was also created in order to initialize 
the call to the validation number \verb+n+.
Notice that we only need to check the divisibility up to the 
square root of the number, because this is the maximum divisor
that a number can have.

\begin{lstlisting}[label=lst_rec,caption=isPrime]
/**
 * @brief isPrime Determines if a number is prime or not recursively
 * @param p number
 * @param n iterator
 * @return
 */
bool isPrime(const unsigned int &p, const unsigned int &n)
{
	if(n<=1)            //Base case: number is always divisible by 1
		return true;
	else if (p%n==0)    //Number divisible by another different than 1 and itself (NOT PRIME)
		return false;
	else
		return isPrime(p, n-1);
}

/**
 * @brief isPrime Determines if a number is prime or not recursively
 * @param p number
 * @return true if the number is prime, false otherwise
 */
bool isPrime(const unsigned int &p)
{
	//We only need to check if the number is divisible by others
	//lower than the square root
	int psqr = sqrt(p);

	return isPrime(p, psqr);
}

\end{lstlisting} 


\section{Arrays}
\subsection{Monodimentional}
In this example, the use of static arrays is very straightforward.
For the dynamic array, I created 2 functions: one for allocating and other for destroying.
Also, a \verb+displayArray+ function was created, which receives a pointer 
to an monodimentional array either
static or dynamic. This was nice because I could reuse the function
for both.

\begin{lstlisting}[label=lst_mono,caption=ArrayExample]
/**
 * @brief allocateArray reserves memory for n integers
 * @param n number of elements
 * @return the allocated array of integers
 */
int* allocateArray(unsigned const int &n)
{
	int* arr = new int[n];
	return arr;
}

/**
 * @brief deleteArray deletes an array dynamically assignated before
 * @param arr the pointer to the array
 */
void deleteArray(int* arr)
{
	if(arr != NULL)
		delete [] arr;
}

/**
 * @brief displayArray is a helper function that allows
 * a pointer to an array to be displayed.
 * @param arr is the pointer to the first element of the array
 * @param N is the number of elements in the array
 */
void displayArray(int *arr, const short &N)
{
	//Iterate through the array and display in stdout
	for(int i=0; i<N; i++)
	{
		cout<<arr[i]<<" ";
	}
	cout<<endl;
}

/**
 * @brief arraysExample Shows the use of the statically and dynamically
 * allocated arrays
 */
void arraysExample()
{
	//Static array creation
	int statarr[NSIZE];

	//Dyanmic array declaration and allocation [Welcome to life!]
	int *dynarr;
	dynarr = allocateArray(NSIZE);

	//Initialize the arrays with the corresponding index
	for(int i=0; i<NSIZE; i++)
	{
		statarr[i] = i;
		dynarr[i] = i;
	}

	//Display the contents of the array
	cout<<"Contents of static array: "<<endl;
	displayArray(statarr, NSIZE);

	cout<<"Contents of dynamic array: "<<endl;
	displayArray(dynarr, NSIZE);

	//Release the memory [I dont need you anymore!]
	deleteArray(dynarr);
}

\end{lstlisting} 

\subsection{Pascal's Triangle}

In this exercice I implemented the same Pascal's triangle done
in the previous lab but taking into account the previous states
that are stored in the positions of the matrix.

\begin{figure}[H]
\centering
%%%%%%%%%%%%% Src code obtained from: http://www.texample.net/tikz/examples/pascals-triangle-and-sierpinski-triangle/
\begin{tikzpicture}[x=13mm,y=9mm]
% some colors
\colorlet{even}{cyan!60!black}
\colorlet{odd}{orange!100!black}
\colorlet{links}{red!70!black}
\colorlet{back}{yellow!20!white}
% some styles
\tikzset{
	box/.style={
		minimum height=5mm,
		inner sep=.7mm,
		outer sep=0mm,
		text width=10mm,
		text centered,
		font=\small\bfseries\sffamily,
		text=#1!50!black,
		draw=#1,
		line width=.25mm,
		top color=#1!5,
		bottom color=#1!40,
		shading angle=0,
		rounded corners=2.3mm,
		drop shadow={fill=#1!40!gray,fill opacity=.8},
		rotate=0,
	},
		link/.style={-latex,links,line width=.3mm},
		plus/.style={text=links,font=\footnotesize\bfseries\sffamily},
}
% Pascal's triangle
% row #0 => value is 1
\node[box=odd] (p-0-0) at (0,0) {1};
\foreach \row in {1,...,5} {
	% col #0 =&gt; value is 1
		\node[box=odd] (p-\row-0) at (-\row/2,-\row) {1};
	\pgfmathsetmacro{\value}{1};
	\foreach \col in {1,...,\row} {
		% iterative formula : val = precval * (row-col+1)/col
			% (+ 0.5 to bypass rounding errors)
			\pgfmathtruncatemacro{\value}{\value*((\row-\col+1)/\col)+0.5};
		\global\let\value=\value
			% position of each value
			\coordinate (pos) at (-\row/2+\col,-\row);
		% odd color for odd value and even color for even value
			\pgfmathtruncatemacro{\rest}{mod(\value,2)}
		\ifnum \rest=0
			\node[box=even] (p-\row-\col) at (pos) {\value};
		\else
			\node[box=odd] (p-\row-\col) at (pos) {\value};
		\fi
			% for arrows and plus sign
			\ifnum \col<\row
			\node[plus,above=0mm of p-\row-\col]{+};
		\pgfmathtruncatemacro{\prow}{\row-1}
		\pgfmathtruncatemacro{\pcol}{\col-1}
		\draw[link] (p-\prow-\pcol) -- (p-\row-\col);
		\draw[link] ( p-\prow-\col) -- (p-\row-\col);
		\fi
	}
}
\begin{pgfonlayer}{background}
% filling and drawing with the same color to enlarge background
\path[draw=back,fill=back,line width=5mm,rounded corners=2.5mm]
(  p-0-0.north west) -- (  p-0-0.north east) --
(p-5-5.north east) -- (p-5-5.south east) --
( p-5-0.south west) -- ( p-5-0.north west) --
cycle;
\end{pgfonlayer}
\end{tikzpicture}
\caption{Pascal Triangle. Source \cite{tikz}} 
\label{fig:pascal}
\end{figure}

This functionality can be divided into 3 parts:
\begin{itemize}
\item Allocation
\item Initialization
\item Calculation
\item Display
\end{itemize} 

Unlike the previous exercice, a generic function that receives
the double pointer to the start of the matrix can not be used for both,
static and dynamic implementations. Since the matrix can only be 
passed as a parameter of a function by explicitly specifying the size
of the columns, it can only be used with static matrices
of that size. 

\begin{lstlisting}[label=pastat,caption=Pascal Static]
/**
 * @brief initialize the matrix mat with NSIZE columns with
 * ones in the first column and zeros in all the other elements
 * @param mat is the matrix to initialize
 */
void initialize(int mat[][NSIZE])
{
	//Initialize the matrix
	for(int i=0; i<NSIZE; i++)
	{
		for(int j=0; j<NSIZE; j++)
		{
			//The first column is all ones
			if(j==0)
				mat[i][j] = 1;
			//The others are initialized to zero
			else
				mat[i][j] = 0;
		}
	}
}

/**
 * @brief calcPascalStatic Calculate the each position as the sum of the previous 2 elements
 * @param mat is the matrix to initialize
 */
void calcPascalStatic(int mat[][NSIZE])
{
	//Calculate the each position as the sum of the previous 2 elements
	//from the row before
	for(int i=1; i<NSIZE; i++)
	{
		for(int j=1; j<NSIZE; j++)
		{
			mat[i][j] = mat[i-1][j] + mat[i-1][j-1];
		}
	}
}

/**
 * @brief displayPascalStatic Displays the values of the given matrix of NSIZE columns
 * @param mat is the matrix to initialize
 */
void displayPascalStatic(int mat[][NSIZE])
{
	//Display the values calculated!
	for(int i=0; i<NSIZE; i++)
	{
		for(int j=0; j<NSIZE; j++)
		{
			printf("%3d ",mat[i][j]);
		}
		printf("\n");
	}
}

/**
 * @brief pascalTriangle1 calculates the elements of the Pascal Triangle
 * obtaining each value from the previous calculated values of the row before.
 */
void pascalTriangleStatic()
{
	int pascal[NSIZE][NSIZE];

	initialize(pascal);
	calcPascalStatic(pascal);
	displayPascalStatic(pascal);
}

\end{lstlisting} 

The dynamic version of the calculation is much more flexible because
it not only allows the user to input the desired last level of the triangle
but also the functions can be invoked for any dynamically allocated integer
matrix.

The code generated by using pointer arithmetic, although it achieves
the same goal as by using he bracket operator \verb+[]+ is much less readable.

\begin{lstlisting}[label=lst_dyn_mat,caption=Pascal Dynamic]
/**
 * @brief allocateMatrix reserves space for an array of n positions were
 * each position points to an array of n integers, thus giving nxn (squared matrix)
 * of reserved space
 * @param n number of rows and columns
 * @return the double pointer to the allocated space
 */
int **allocateMatrix(const unsigned int &n)
{
	int **mat = new int*[n];

	for(unsigned int i=0; i<n; i++)
		*(mat+i) = new int[n];

	return mat;
}

/**
 * @brief deleteMatrix releases the allocated space for a matrix mat
 * of n columns
 * @param mat matrix to be released
 * @param n number of rows
 */
void deleteMatrix(int ** mat, const unsigned int &n)
{
	//For each row of the matrix delete the pointer array
	for(unsigned int i=0; i<n; i++)
	{
		delete [] *(mat+i);
	}
}

/**
 * @brief initializeMatrix initializes the matrix mat with n rows and n columns.
 * Assigns ones in the first column and zeros in all the other elements
 * @param mat is the matrix to initialize
 * @param n number of rows
 */
void initializeMatrix(int ** mat, const unsigned int &n)
{
	//Initialize the matrix
	for(unsigned int i=0; i<n; i++)
	{
		for(unsigned int j=0; j<n; j++)
		{
			//The first column is all ones
			if(j==0)
				*(*(mat + i) + j) = 1;
			//The others are initialized to zero
			else
				*(*(mat + i) + j) = 0;
		}
	}
}

/**
 * @brief calcPascalStatic Calculate the each position as the sum of the previous 2 elements
 * @param mat is the matrix to initialize
 */
void calcPascal(int ** mat, const unsigned int &n)
{
	//Calculate the each position as the sum of the previous 2 elements
	//from the row before
	for(unsigned int i=1; i<n; i++)
	{
		for(unsigned int j=1; j<n; j++)
		{
			*(*(mat + i) + j) = *(*(mat + i - 1) + j) + *(*(mat + i - 1) + j - 1);
		}
	}
}

/**
 * @brief displayPascalStatic Displays the values of the given matrix of NSIZE columns
 * @param mat is the matrix to initialize
 */
void displayPascal(int ** mat, const unsigned int &n)
{
	//Display the values calculated!
	for(unsigned int i=0; i<n; i++)
	{
		for(unsigned int j=0; j<n; j++)
		{
			printf("%3d ",*(*(mat + i) + j));
		}
		printf("\n");
	}
}

/**
 * @brief pascalTriangleDynamic calculates the elements of the Pascal Triangle
 * obtaining each value from the previous calculated values of the row before.
 */
void pascalTriangleDynamic(unsigned const int &n)
{
	int **mat;

	mat = allocateMatrix(n);
	initializeMatrix(mat, n);
	calcPascal(mat, n);
	displayPascal(mat, n);
	deleteMatrix(mat, n);
}
\end{lstlisting} 

\section{Multidimensional arrays as function parameters}

The use of static matrices as parameters was already used and explained in
the previous section. 

\begin{lstlisting}[label=lst_mat_par,caption=Matrix Parameters]
/**
 * @brief multMatrix multiplies a 3x3 static matrix C=AxB
 * @param A Matrix 1
 * @param B Matrix 2
 * @param C Result matrix
 */
void multMatrix(int A[][SQRMAT], int B[][SQRMAT], int C[][SQRMAT])
{
	for(unsigned short i=0; i<SQRMAT; i++)
	{
		for(unsigned short j=0; j<SQRMAT; j++)
		{
			for(unsigned short k=0; k<SQRMAT; k++)
			{
				C[i][j] += A[i][k] * B[k][j];
			}
		}
	}
}

/**
 * @brief displayMatrix Displays a 3x3 matrix
 * @param C matrix to display
 */
void displayMatrix(int C[][SQRMAT])
{
	for(unsigned short i=0; i<SQRMAT; i++)
	{
		for(unsigned short j=0; j<SQRMAT; j++)
		{
			cout<<C[i][j]<<" ";
		}
		cout<<endl;
	}
}
/**
 * @brief matMultCaller controlls the call to the matMult function
 */
void matMultCaller()
{
	int A[][3] = {{1, 2, 3}, {4, 5, 6}, {7, 8, 9}};
	int B[][3] = {{1, 2, -1}, {1, 2, -1}, {1, 2, -1}};
	int C[3][3] = {{}};

	multMatrix(A,B,C);

	displayMatrix(A);
	cout<<"TIMES "<<endl;
	displayMatrix(B);
	cout<<"EQUALS "<<endl;
	displayMatrix(C);
}
\end{lstlisting} 

\bibliographystyle{plain}
\bibliography{biblio}

\end{document}
