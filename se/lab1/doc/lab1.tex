\documentclass{article}

\usepackage{mathtools}
\usepackage{graphicx}
\usepackage{subfig}
\usepackage{verbatim}
\usepackage{algpseudocode}
\usepackage{natbib}
\usepackage{url}
\usepackage{listings}
\usepackage[spanish]{babel}
\usepackage[utf8x]{inputenc}
\usepackage{float}
\usepackage{array}
\usepackage{booktabs}
\usepackage{amssymb}
%\setcounter{MaxMatrixCols}{16}


\begin{document}
\title{Report Lab 1}
\date {\today}
\author{Tatiana Lopez Guevara}
\maketitle


\section{Variables}
\subsection{Local vs Global}
A global variable can be accessed from any part of the program and will 
be alive during the execution of the whole program.
A local variable is only accessible inside the scope its 
declared and will be alive during the execution of the function or scope it lives in.

The section \ref{sec:fibo} shows the use of the global variable $\varphi$ that holds 
the known value of the golden ratio. Inside the function \verb|golden_ratio| 
the value $err$ is a local variable which is only accessible inside that function.

\subsection{Elementary functions}
In this section I created the functions \verb|mean, min, max| which received 2 
integer parameters. In this case the input variables do not need to be modified
and therefore I learned how to help a little bit the compiler by explicitly 
specifying the read only property with \verb+ const + keyword and also by not 
creating a copy by passing it a reference with \verb+ & +. 

Also, in the function \verb+mean+ instead of dividing by 2, a multiplication
for 0.5 was used because it uses less clock cycles in the execution.

\section{Combination}
\subsection{Factorial}
In this function we noticed 2 things besides the optimizations mentioned 
int the previous section:
\begin{itemize}
\item The Fibonacci is only defined for values in $\mathbb{N}^0$ and therefore
is better to declare it as unsigned. Due to the extremely fast growing of the
factorial function we declared the type as \verb+short+.
\item The return type must be able to hold a very large number. At the begining
we declared it as \verb+long long+ but then we realized that it overflowed with
the factorial of numbers greater than 21 and therefore we had to change it to
\verb+double+.
\end{itemize}

\subsection{Combinations}
In order to follow the KISS phylosphy, I just reused the 
\verb+factorial+ function created in the previous step to calculate $n!, k!$
and $(n-k)!$ used in the formula.

The number of combinations of a lottery game is $C^{49}_{6}$ and 
the program gave the result 1.39838e+07.

\subsection{Combinations with repetitions}
Same as before. I just reused the \verb+combinations+ function to get
the term:
\begin{equation*}
\binom{n+k-1}{k}
\end{equation*} 

\subsection{Permutations}
In this function I reused the two previous functions to 
get the value $P_5^{54} = 3.79501e+08$ from
the formula:

\begin{equation*}
P_k^n = \frac{n!}{(n-k)!} 
\end{equation*} 


\section{Fibonacci}
\label{sec:fibo}
The Fibonacci function has a recursive definition, however the implementation
realized here was not, first because for printing each one of the terms it was 
more easy and second because it is faster.

\subsection{Golden Ratio}
The same logic of the previous section was implemented in this function and the
ratio between the current value and the previous value was calculated and printed
at each step. When the variable of the error and the rate was declared as \verb+float+
the loop went to an infinite cycle because of precision \cite{yohan1} when calculating it with
the $\epsilon = 10^{-9}$. The results obtained with the \verb+double+ declaration where:

\begin{itemize}
\item $\epsilon=10^{-6}$ stopped with $F_{16}=987$ and $F_{17}=1597$. 
\item $\epsilon=10^{-9}$ stopped with $F_{23}=28657$ and $F_{24}=46368$. 
\item $\epsilon=10^{-12}$ stopped with $F_{30}=832040$ and $F_{31}=1346269$. 
\item $\epsilon=10^{-15}$ stopped with $F_{37}=24157817$ and $F_{38}=39088169$. 
\end{itemize} 

\section{Pascal's Triangle}
One of the most simple and effective ways to print the Pascal's triangle
is using arrays to calculate the current value only through sums and
based only on 2 previous factos. 
However, this was implemented for Lab2 and here I just used
another simple approach by just reusing the \verb+combinations+ function
for each calculation.

\section{Other things}
\subsection{String}
I used 5 ways to create strings: the default constructor, passing a string,
assigning a value with the \verb+ = + operator, and repeating a character 
constructor \cite{cpp}. 
The empty string was then assigned to the concatenated value of
the first 2 strings using the \verb|+| operator. The last string was
used to play with the \verb+append+ and \verb+insert+ functions. 
All values were shown in the output ussing the \verb+<<+ operator and
\verb+c_str()+ 

\subsection{Main}
The parameters of the main are used for accessing the information
of the parameters passed into the program on the invocation.
More specifically, \verb+argc+ is the number of parameters passed 
where the 1st one is the program invocation by itself.
\verb+argv+ is a vector of \verb+argc+ possitions, each one corresponding
to the value of the parameters for the program.

To illustrate this, at the begining of the \verb+main+, I reused the proposed
code in the last point of the lab.

\bibliographystyle{plain}
\bibliography{biblio}

\end{document}
