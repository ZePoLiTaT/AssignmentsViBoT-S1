\documentclass{article}


 % Required for inserting code snippets
\usepackage{mathtools}
\usepackage{graphicx}
\usepackage{subfig}
\usepackage{verbatim}
\usepackage{algpseudocode}
\usepackage{natbib}
\usepackage{url}
\usepackage{listings}
\usepackage{float}
\usepackage{array}
\usepackage{booktabs}
\usepackage{amssymb}
%\setcounter{MaxMatrixCols}{16}

%%%%%%%%%%%%%%%%%%%%%%%% PAINTING LIBRARY
\usepackage{pgf}
\usepackage{tikz}
\usetikzlibrary{arrows,automata}
\usepackage[latin1]{inputenc}
\usepackage[upright]{fourier}
\usetikzlibrary{matrix}
\usepackage{fullpage,amsmath}
\usepackage{times}
\usepackage{geometry}
\usetikzlibrary{mindmap,backgrounds}

%%%%%%%%%%%%%%%%%%%%%%%% THNX TO OZAN FOR THE CODE HEADER ;)
\usepackage{color}
\usepackage{xcolor}
\usepackage{listings}

\usepackage{courier}
\definecolor{DarkGray}{rgb}{0.43,0.35,0.35} % Comment color
\lstset{
	language=C++,								% choose the language of the code
		basicstyle=\footnotesize\ttfamily,  % the size of the fonts that are used for the code
		numbers= left,						% where to put the line-numbers
		numberstyle=\tiny,					% the size of the fonts that are used for the line-numbers
		stepnumber=1,						% the step between two line-numbers. If it is 1 each line will be numbered
		numbersep=10pt,						% how far the line-numbers are from the code
		backgroundcolor=\color{white},		% choose the background color. You must add \usepackage{color}
		keywordstyle=\color{DarkGray}\bf,
	showspaces=false,						% show spaces adding particular underscores
		showstringspaces=false,				% underline spaces within strings
		showtabs=false,						% show tabs within strings adding particular underscores
		keywordstyle=\color{DarkGray}\bf,
		stringstyle=\color[rgb]{0.627,0.126,0.941},
		xleftmargin=17pt,
		framexleftmargin=17pt,
		framexrightmargin=5pt,
		framexbottommargin=4pt,
		%frame=b,         
		frame=single,					% adds a frame around the code
		tabsize=2,						% sets default tabsize to 2 spaces
		captionpos=t,					% sets the caption-position to bottom (t=top, b=bottom)
		breaklines=true,				% sets automatic line breaking
		breakatwhitespace=false,		% sets if automatic breaks should only happen at whitespace
		escapeinside={\%*}{*)}          % if you want to add a comment within your code
}

\usepackage{caption}
\DeclareCaptionFont{white}{\color{white}}
\DeclareCaptionFormat{listing}
{
	\colorbox[rgb]{0.83, 0.85, 0.88}
	{\parbox{\dimexpr\textwidth-8\fboxsep\relax}{#1#2#3}}
}

\captionsetup[lstlisting]
{
	format=listing,
		labelfont=white,
		textfont=white, 
		singlelinecheck=false, 
		margin=0pt, 
		font={bf,footnotesize}
}

\usepackage{textcomp}

%%%%%%%%%%%%% WATERMARKS  %%%%%%%%%%%%%%%%%
%\usepackage{eso-pic}
%\AddToShipoutPicture{
%    \includegraphics[width=1.74\textwidth,natwidth=1223,natheight=121]{figs/lowImg2.png}}

\DeclarePairedDelimiter\abs{\lvert}{\rvert}


%%%%%%%%%%%%% PARAGRAPHS %%%%%%%%%%%%%%%%%
%No ident in new paragraph
\usepackage[parfill]{parskip}
%% Add a space of 1.5 lines between paragraphs
\parskip=1.1\baselineskip

%%%%%%%%%%%%% DEFAULT FONT %%%%%%%%%%%%%%%
%Sans-serif font by default
%\renewcommand{\familydefault}{\sfdefault}
%\usepackage{mathptmx}

%\usepackage{bookman}
%%%%\usepackage[default]{droidserif}
%\usepackage[T1]{fontenc}
%\usepackage{lxfonts}
%\usepackage{tgadventor}
%\renewcommand*\familydefault{\sfdefault} %% Only if the base font of the document is to be sans serif
%\usepackage[T1]{fontenc}


\newcommand{\specialcell}[2][c]{%
  \begin{tabular}[#1]{@{}c@{}}#2\end{tabular}}

\usepackage{float}



\begin{document}
\title{Infrared Imaging (Lab 2)}
\date {\today}
\author{Klemen Istenic\\Tatiana Lopez G}
%\author{Tatiana Lopez Guevara}
\maketitle

\section{Altair Software}
In the first assignment of the lab we used the IR 
camera to check if there is any defect on the electronic circuit 
provided.  Our aim was to capture the circuit with IR camera 
and observe if there is any increase in the temperature. This 
would imply a possible defect.
 
\begin{figure}[H]
\centering
\begin{minipage}[b]{0.45\linewidth}
\includegraphics[width=1.0\textwidth,natwidth=100,natheight=100]{../results/klemen/CircuitBefore001.PNG}
\end{minipage}
\quad
\begin{minipage}[b]{0.45\linewidth}
\includegraphics[width=1.0\textwidth,natwidth=100,natheight=100]{../results/klemen/CircuitAfter001.PNG}
\end{minipage}
\caption{IR of circuit OFF and ON}
\label{fig:p1}
\end{figure}

Figure \ref{fig:p1} shows the electrical circuit before and after we apply the 
voltage. On the right photo we can clearly see that the temperature has 
significantly increased in two spots indicating two defects.

Because we did not 
calibrate the camera before taking the measurements, we could not calculate 
the temperatures, but instead we provided all informations in DL (
digital level) units.

With the provided software (Altair software) we also measured 
mean digital level value  of both of the spots of defect, 
as well as mean digital level of the whole circuit
 (table \ref{tb:tb1}). 

\begin{table}[H]
\centering
\begin{tabular}{|c|c|c|}
\hline
&Before &	After \\
\hline
Spot A – max DL	&	6846	&15755\\
Spot B – max DL &	6810	&9983\\
Mean DL of whole circuit &	6824	&6991\\
\hline
\end{tabular}
\caption{Measurements Before and After (*DL - digital level)}
\label{tb:tb1}
\end{table} 

We also measured the profile before and after on the line 
indicated on Figure \ref{fig:p1}. On the first graph (taken 
before) we can see that the temperature is somehow constant, 
with an exception of the middle part corresponding to the hole in 
the middle of the circuit. The second graph on the right of 
the figure (taken after
) shows the huge increase of temperature on the defected spot, 
showing as a spike on the graph. 

\begin{figure}[H]
\centering
\begin{minipage}[b]{0.48\linewidth}
\includegraphics[width=1.0\textwidth,natwidth=100,natheight=100]{../results/klemen/CircuitBeforeProfile.PNG}
\end{minipage}
\quad
\begin{minipage}[b]{0.48\linewidth}
\includegraphics[width=1.0\textwidth,natwidth=100,natheight=100]{../results/klemen/CircuitAfterProfile.PNG}
\end{minipage}
\caption{Profile Before and After}
\label{fig:p1b}
\end{figure}



\section{Camera Calibration}


For the calibration we used the hot plate where we gradually increased the 
temperature and at the same time acquired the information with the IR camera.
We measured the exact temperature of the hot plate using a digital thermometer
and recorded the digital levels given by the Altair software (Table \ref{tb:calib}).

In figure \ref{fig:calib} we plotted the results and the fitted "calibration" 
line. We can see that the measurements are a little bit unstable which is due to the 
high resolution of the camera which can record even small fluctuations in
the environment. 

\begin{table}[H]
\centering
\begin{tabular}{|c|c|c|c|c|c|c|c|}
\hline
$T^\circ$ & 31.6 & 39.4 & 45.8 & 49.7 & 55.1 & 59.3 & 65.1 \\
DL& 8213 & 8615 & 8853 & 9329 & 9623 & 10067 & 10748 \\ 
\hline
\end{tabular}
\caption{Calibration Measurements}
\label{tb:calib}
\end{table}


\begin{figure}[H]
\centering
\begin{minipage}[b]{0.8\linewidth}
\includegraphics[width=1.0\textwidth,natwidth=100,natheight=100]{../results/klemen/GraphCalibration.PNG}
\end{minipage}
\caption{Calibration plot}
\label{fig:calib}
\end{figure}
The experiment could be better if we could control the  
temperature of the hotplate electronically avoiding possible errors
when measuring the temperature itself.

\section{Coefficients Estimation}
As we saw in class, the emissivity 
can be defined as the ratio between the luminance of a real material and a black body.

That is why, we first needed to take the measurement of the luminance of the black body.
Since there is not such thing as a perfect black body, we had
to approximate its value using a black paper. Table \ref{tb:tb2a} shows
the obtained results.

\begin{table}[H]
\centering
\begin{tabular}{|c|c|c|}
\hline
Black velvet (DL) & 7053 \\
Wood (DL) & 6947\\
\hline
Emissivity $\epsilon$ & 0.98497\\ 
\hline
\end{tabular}
\caption{Wood Emissivity}
\label{tb:tb2a}
\end{table} 

By definition, the transmission coefficient is the percentage 
that gets through the object. Therefore, what we did was to take measurements
from the black body, and then place the object in front and take the measurement
again (Table \ref{tb:tb2b}). The ratio between them gave us the value for
transmissivity.

\begin{table}[H]
\centering
\begin{tabular}{|c|c|c|}
\hline
Black velvet (DL)& 7071 \\
PVC (DL) & 6937\\
\hline
Transmissivity $\tau$ & 0.981 \\ 
\hline
\end{tabular}
\caption{PVC Transmissivity}
\label{tb:tb2b}
\end{table} 

To calculate the reflection coefficient of the copper we should have 
taken the measurements of the reflection of the black body from copper's surface.
However, during our lab, instead of the black body we made a mistake
and instead, we used one of our fingers because
the reflection was clearly seen in the image and that deceived us. 
While doing the report
we realized that actually we should have used the black paper because it's emissivity
coefficient is close to 1 in all directions and therefore the calculated
value would have been correct.

\begin{figure}[H]
\centering
\begin{minipage}[b]{0.5\linewidth}
\includegraphics[width=1.0\textwidth,natwidth=100,natheight=100]{../results/klemen/copper.PNG}
\end{minipage}
\caption{Reflection on the surface of the copper}
\label{fig:calib}
\end{figure}

\section{Active Thermography}

The idea of this section was to see the how an IR camera can 
be applied in NDT or Non Destructive Thermography. A halogen lamp 
connected to a sinusoidal voltage generator was used.

\subsection{First Experiment}

In this section we applied the principles of active thermography
in order to analyze a painting. 

\begin{figure}[H]
\centering
\begin{minipage}[b]{0.4\linewidth}
\includegraphics[width=1.0\textwidth,natwidth=100,natheight=100]{../results/img/at1_middle2.PNG}
\end{minipage}
\quad
\begin{minipage}[b]{0.4\linewidth}
\includegraphics[width=1.0\textwidth,natwidth=100,natheight=100]{../results/img/at1_initial.PNG}
\end{minipage}
\caption{IR image after heating with halogen lamp connected to a sinusoidal generator}
\label{fig:at1}
\end{figure}

With visual inspection of the IR image, we could not see a lot of details in the painting. 
However, after heating it with the sinusoidal generator, we observed the
shape of an elongated figure covering the middle part of the board which
could not be seen before. A few seconds after stopping the heating
source, we could notice that the abnormal layer of painting
took more time to cool down which made it more clear revealing a humanoid
figure and a ring (figure \ref{fig:at1}). This leads us to think 
that such shape was part of a previous painting which then 
was used as a canvas for the new painting.

The principle here is, as we saw in class, that the defects, or in this 
case the other layer of painting, take more time to cool down and therefore
can be detected by the IR camera without destroying the object.

\subsection{Second Experiment}

Here we analyzed the flat-bottom-holes object provided. This structure
was placed in front of the IR camera with the holes opposite to it.
Initially the holes were not visible either by visual inspection
nor by the IR camera.

Then we started heating the object with the halogen lamp connected to
the voltage generator to 1Hz. In this case, the holes from the left
started to appear first. After some more time, all the holes were totally
visible through the IR camera (figure \ref{fig:at2}).

\begin{figure}[H]
\centering
\begin{minipage}[b]{0.7\linewidth}
\includegraphics[width=1.0\textwidth,natwidth=100,natheight=100]{../results/img/holes001.PNG}
\end{minipage}
\caption{Flat-bottom-holes after heating}
\label{fig:at2}
\end{figure}

The difference in time that took the holes to appear was because
of the different depth of the holes. The holes that were deeper (on the left side)
had less material in front of them, so it took less time to warm it 
(the material).


\subsection{Conclusions}
As we could experience here, the principles of active thermography
can be applied in a great variety of fields. In the first experiment
we saw the application in the field of Art were IR NDT can be applied
to identify fake paintings, reused canvas, deterioration detection, etc.
In the second example, we saw the application in the field of 
detecting thickness of materials, finding defects, etc. 

% \bibliographystyle{plain}
% \bibliography{biblio}
\end{document}
